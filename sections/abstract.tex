% !Mode:: "TeX:UTF-8"
% !TEX program  = xelatex
\begin{中文摘要}{人体动作识别,人体骨架,微多普勒,雷达散射截面}
  人体动作识别在非视距(non-line-of-sight)环境中检测、跟踪和监控人类活动起着非常重要的作用。
  人体的四肢和躯干在运动时,会产生独特的微多普勒特征,这些微多普勒特征为识别与分类人体的运动特征提供了可能。


  本文对人体的动作识别与分类完成了如下几个工作:
  
  
  (1) 通过模拟的方法,从计算机动画数据中提取出人体的三维骨架数据以及人体的运动特征,并且使用MATLAB对人体运动进行建模。\par


  (2)将人体的运动分解为平移和旋转两个部分,并且推导人体局部变换矩阵,计算人体运动每一时刻的雷达散射截面(RCS: Radar Cross Section),对人体不同运动产生的微多普勒特征进行定性分析。\par


  (3)介绍神经网络相关原理,通过卷积神经网络(CNN: Convolutional neural network)算法,对人体的动作产生的微多普勒特征进行识别与分类,分析分类的准确率。\par

\end{中文摘要}

\begin{英文摘要}{Human motion recognition, human skeleton, micro-Doppler, Radar Cross Section}
  Human motion recognition plays an important role in detecting, tracking and monitoring human activities in non-line-of-sight environment.
  During the movement, the human limbs and torso will result in unique micro-Doppler characteristics. 
  These micro-Doppler features make it possible to recognize and classify human motion features.


  In this paper, the following work has been completed for human motion recognition and classification:


  (1)Through simulation method, the three-dimensional skeleton data of human body and the motion characteristics of human body are extracted from the computer animation data, 
  and the human motion is modeled by MATLAB.


  (2) Decompose the motion of human body into two parts: translation and rotation. 
  Deduce Local Transformation Matrix of human body and calculate the Radar Cross Section at each moment of human motion. 
  Finally, the micro-Doppler characteristics produced by different human movements are qualitatively analyzed. 


  (3) Introduces the relevant principles of neural network, 
  identifies and classifies the micro Doppler features generated by human actions through the convolutional neural network (CNN) algorithm, 
  and analyzes the accuracy of classification.


\end{英文摘要}
